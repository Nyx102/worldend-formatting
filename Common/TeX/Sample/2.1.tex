Lillia Asplay knew emptiness.

Not as an abstract concept but through experience.

Back then—four years ago, when she was ten years old—that emptiness was inside her.

\icon

Lillia was an honest girl.

She listened to her elders and fulfilled her duties with a smile on her face.

She was the daughter of the king in Dione, Kingdom of Knights. She was fourteenth in line for the throne. Dione itself was a small, pastoral country anyway—it was a stranger to quarrels over throne succession.

As a symbol of such a peaceful country, everyone wanted her to be like a doll, one who smiled brightly, innocently, as if she was ignorant of everything. And since she had always been extraordinarily intelligent, she understood that well. Not only did she understand, but she accepted it.

If smiling meant the adults around her could find some measure of peace and salvation, then so be it. She thought that so long as the muscles in her cheeks held out, she would smile as much as she could.

But this should not be misunderstood—Lillia never thought of these days as miserable. Both her parents were busy, but they loved their daughter, and the high-ranking nobles and the strong knights from the Order were all generally kind people. Lillia’s smile was not entirely an act; it was perhaps more apt to say that she was shouldered with a task that made use of what came naturally to her.

But at age nine, her whole world changed.

There was a monstrous race called the elves. They looked like warped and twisted decaying trees and traveled together in nimble swarms, like some sort of sick joke. The elves were classified as spirit-type monsters, which meant they were supposed to be highly intelligent and possessing advanced technology. However, since they were unable to communicate with humans, there was no confirmation of such claims. The longevity of this race meant they had a storied history, and they still used technology from ancient times. Official military documents and the like often referred to the elves as “the elder spirits” or “the elder kind.” Although they rarely emerged from the “murk-wood” they called home, they would sometimes expand the territory of the murk-wood themselves, coming together in great swarms and attacking human domains.

A hundred or so gloom elves fell upon the Dione territory like a plague.

The attack came before dawn. Just before the kitchen smoke started rising from the chimneys of people’s homes, a completely different kind of flame engulfed every corner of the city. The guards and knights stationed around the city for an emergency were decimated; there was hardly anything they could do against a horde of absurdly strong monsters launching a surprise attack.

The country was wiped off the map.

The few who survived had escaped through a secret passageway with the help of a loyal retainer—and among them was the young Princess Lillia.

The story up until that point was rather well-known. Most of the people who heard the tale believed this was the moment when Lillia Asplay lost everything.

In one sense, they were correct. Lillia lost a great many things.

On the other hand, they were wrong. Lillia’s losses would come quite some time later.

Afterward, Lillia was treated like a tragic heroine everywhere she went.

From that day forward, people wanted this girl to play a part unlike any she’d had before.

Everything she loved had been lost, robbed from her by a horde of the wicked. She had watched it all disappear into the flames with her own eyes—things precious to her, things she didn’t particularly care for, things she wanted to hold on to forever, things she had hoped would vanish. Everything was reduced to ash all the same, no more, no less.

She should have been sad.

She should have been in pain.

She should have lost hope.

She should have been angry.

She should have hated them for it.

Everyone wanted this princess of a lost country to be the protagonist of a tragedy. They wanted her to be this poor, sad little girl. It was like looking out onto the snowy landscape from inside a warm house. To all those who believed they were not unhappy, looking upon the unfortunate was a kind of mild entertainment.

And Lillia was an obedient girl.

She listened to her elders and fulfilled her duties with a smile on her face.

She was sad for them. She was in pain for them. She lost hope for them. She was angry for them. She hated for them. She brought to life what all the adults around her had been hoping for, all while wearing a withered, lifeless smile plastered on her face.

One day, in the darkness, Lillia realized something.

Was she truly sad? Was she truly in pain? Had she lost hope? Was she angry? Did she hate anyone?

She certainly possessed those feelings inside her. But she didn’t know where they came from.

On that day when she stood there, what had nine-year-old Lillia Asplay thought as she watched everything burn away?

She couldn’t remember.

Everyone else’s expectations about how she was supposed to feel and think, repeated to her over and over, had overwritten her memories and feelings of that day.

Once she came to this realization, the girl who had always worked to be what was expected of her forgot who she originally was.

\icon

A year passed.

Lillia was ten.

“Wait here,” an old man said to her in a little hut. Then, along with another sturdy-looking old man, he left the shack.

She could have obeyed the command and waited there. It wasn’t like she had anything in particular she wanted to do. She was already used to sitting, well-behaved and pleasing everyone. Suppressing her own feelings to avoid boredom was her forte, after all. No matter how many hours…or how many days, even, she could have sat there, obediently waiting, the entire time.

However.

This one time, for some reason, she succumbed to temptation.

The girl stepped out into the remote, empty forest.

When people do something they normally try not to do, they wind up seeing things they normally try not to see.

In a small clearing in the woods, there was a boy, about ten years old, waving around a stick.

She probably wasn’t imagining the steam that she saw wafting from his body. Despite how cold it was outside, there were more signs that the boy must have been moving about furiously for a long time; he was drenched in sweat, and even the ground beneath his feet looked well-trodden.

In many ways, he was being much too enthusiastic for just playing pretend swordsman.

Lillia hid herself behind a tree and decided to watch him for a bit.

His grip was light, but in contrast, his steps were long and deep. The center of gravity in his basic stance was oddly high, yet his stance when he delivered the blow was rather low. As she watched the boy spin around like a badly made top, she slowly started to see why he moved so strangely.

He was probably trying to train in all sorts of weapons at one time.

From just a glance, it looked like play sword fighting that was a little on the advanced side. His movements essentially resembled fencing. But on closer inspection, certain things changed slightly in the intervals between weapons. He would alter his grip just a touch, replicating a multitude of weapons with just one stick—or rather, she could see through his movements that he was striving to reach a point where he could pull off such a feat.

But it was a pity that, after all was said and done, the boy was incompetent.

His training was probably focused on the way he held his fingers, which controlled the intervals between weapons. But the movements in the boy’s hands were clearly awkward. Same for the way he carried himself. Not only did he lack strength and weight in his physique, but in order to make a powerful attack, he would have to skillfully “drop” his heightened center of gravity onto his point of attack somehow. But what little power this boy did have was mostly escaping through the bottom of his shoes into the dirt. If he couldn’t move any lighter, like he was dancing above the clouds, then his training would never amount to anything more than just slightly advanced swordplay.

The more Lillia watched, the more frustrated she became.

The more her frustration mounted, the more irritated she became.

And yet, somehow, she couldn’t look away.

Her vision blurred. She realized that, for some reason, tears were pooling in her eyes. She didn’t understand why, but they would spill over if she left them alone. She didn’t want that to happen, so as she kept her eyes on the boy, she wiped away the moisture, one side at a time.

Suddenly, the boy slipped.

Oh, she thought.

Oh, was the expression she saw on the boy’s face.

He flipped over as his shoes drew a clean arc in the air. There was a loud thud when his back hit the earth. That must have hurt. The boy hadn’t just tripped and fallen—he’d practically thrown himself onto the ground. But given how soft the dirt was, it was unlikely he’d actually injured himself.

“—Owwwww!” the boy hollered.

With a cry, he disguised his frustration at his body for not moving the way he wanted it to.

Most likely, his exhausted limbs had already been begging for a respite. He lay on the ground with his arms and legs splayed out, gazing far up at the blue sky—

“…”

—and noticed her.

Their eyes met.

He probably hadn’t even imagined there would be someone watching. There was a momentary flash of surprise in his eyes before it slowly transformed into embarrassment.

“Who…who are you?!”

His cheeks were flushed, but that was pretty normal considering he’d just had an intense workout. Embarrassed and flustered, the boy leaped to his feet. He brushed off the dirt that clung to his clothes, picked up the stick that had fallen from his hands, and took a thunder focus stance, almost like he had not flipped over just now.

“W-were you watching me?!”

Yep, every second of it.

…The girl’s honest response almost left her mouth, but she quickly held her tongue.

She probably shouldn’t say that to him. That would be a terrible response, one that would hurt the pride of a boy who had (what appeared to be) very little to begin with. Her ten years of life experience, as a sheltered princess as well as a tragic heroine, told her not to say it.

But nevertheless, it didn’t seem like things would turn out well if she stayed quiet. The boy was shooting a reproachful look right in her direction. He wanted some kind of reaction.

She had to say something. Her panic was dulling her young sense of judgment.

The words that came to mind immediately slipped off the tip of her tongue.

“You—”

“…You?”

“You suck.”

In that moment, time froze.

The girl could hear the boy’s pride not only being hurt but shattering into a million pieces.

That was the little girl’s—Lillia Asplay’s—recollection of when she first met the boy who would become her fellow student of the sword.

And it was that precise moment that triggered the boy—Willem Kmetsch—who was always kind and generous to everyone he knew, to treat his fellow disciple Lillia as the sole exception.